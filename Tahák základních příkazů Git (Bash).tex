\documentclass[a4paper,12pt]{article}

\usepackage[czech]{babel}
\usepackage[utf8]{inputenc}
\usepackage[T1]{fontenc}
\usepackage{array}
\usepackage{geometry}
\usepackage{longtable}
\usepackage{hyperref}
\usepackage{xcolor}

\geometry{margin=2.5cm}

\setlength{\parskip}{0.5em}
\setlength{\parindent}{0pt}

\begin{document}

\title{\textbf{Tahák základních příkazů Git (Bash)}}
\author{{\small © 2025 ~Vaňha Jiří~(\textregistered). Distribuce povolena pro studijní účely.}}

\maketitle

\section*{Základní práce s repozitářem}

\begin{longtable}{>{\ttfamily}p{0.35\linewidth} p{0.6\linewidth}}
\hline
git init & Inicializuje nový lokální Git repozitář. \\
git clone <url> & Naklonuje vzdálený repozitář (např. z GitHubu). \\
git status & Zobrazí stav pracovního adresáře a změn. \\
git add <soubor> & Přidá soubor do tzv. staging area (příprava na commit). \\
git add . & Přidá všechny nové i upravené soubory. \\
git commit -m "popis" & Uloží změny do historie s popisem. \\
git log & Zobrazí historii commitů. \\
git log --oneline & Kratší výpis commitů (hash + zpráva). \\
git diff & Zobrazí rozdíly oproti poslednímu commitu. \\
\hline
\end{longtable}

\section*{Práce s větvemi}

\begin{longtable}{>{\ttfamily}p{0.35\linewidth} p{0.6\linewidth}}
\hline
git branch & Zobrazí všechny lokální větve. \\
git branch <název> & Vytvoří novou větev. \\
git checkout <název> & Přepne se do dané větve. \\
git switch <název> & Alternativa k checkout pro přepnutí větve. \\
git checkout -b <název> & Vytvoří novou větev a rovnou se do ní přepne. \\
git merge <větev> & Sloučí větev do aktuální. \\
git branch -d <větev> & Smaže větev (pokud je sloučená). \\
git branch -D <větev> & Násilně smaže větev (i když není sloučená). \\
git log master..bug --oneline & Zobrazí commity, které jsou pouze ve větvi \texttt{bug}. \\
\hline
\end{longtable}

\section*{Vzdálený repozitář (GitHub, GitLab)}

\begin{longtable}{>{\ttfamily}p{0.35\linewidth} p{0.6\linewidth}}
\hline
git remote -v & Zobrazí propojené vzdálené repozitáře. \\
git remote add origin <url> & Přidá vzdálený repozitář. \\
git push -u origin master & Nahraje větev \texttt{master} na server. \\
git push & Nahraje všechny změny. \\
git pull & Stáhne a sloučí změny z remote repozitáře. \\
git fetch & Stáhne aktualizace bez automatického sloučení. \\
git push origin --delete <větev> & Smaže vzdálenou větev. \\
\hline
\end{longtable}

\section*{Práce se změnami}

\begin{longtable}{>{\ttfamily}p{0.35\linewidth} p{0.6\linewidth}}
\hline
git restore <soubor> & Vrátí soubor do posledního commitu. \\
git restore --staged <soubor> & Odebere soubor ze stagingu. \\
git reset --hard & Vrátí vše do stavu posledního commitu (pozor!). \\
git stash push -m "dočasná záloha" & Uloží rozpracované změny bokem. \\
git stash list & Zobrazí uložené stash zálohy. \\
git stash apply stash@{0} & Obnoví konkrétní stash. \\
git diff master..bug & Porovná rozdíly mezi dvěma větvemi. \\
\hline
\end{longtable}

\section*{Patche a archivace}

\begin{longtable}{>{\ttfamily}p{0.35\linewidth} p{0.6\linewidth}}
\hline
git diff master..bug > bug.patch & Vytvoří patch soubor se změnami. \\
git apply bug.patch & Aplikuje patch zpět do aktuální větve. \\
git format-patch master..bug -o ./patches & Vytvoří sadu patchů po commitech. \\
git am ./patches/*.patch & Aplikuje všechny patche i s historií. \\
\hline
\end{longtable}

\section*{Užitečné informace}

\begin{longtable}{>{\ttfamily}p{0.35\linewidth} p{0.6\linewidth}}
\hline
git config --list & Zobrazí aktuální konfiguraci. \\
git config user.name "Jméno" & Nastaví jméno pro commity. \\
git config user.email "email@example.com" & Nastaví e-mail. \\
git show <hash> & Zobrazí detailní informace o commitu. \\
git reflog & Zobrazí i smazané/ztracené commity. \\
\hline
\end{longtable}

\bigskip
\textit{Poznámka:} Tento tahák obsahuje nejčastěji používané příkazy pro práci s Gitem v prostředí Bash. Lze jej rozšiřovat o pokročilé příkazy jako \texttt{rebase}, \texttt{cherry-pick} nebo \texttt{bisect}.

\end{document}
