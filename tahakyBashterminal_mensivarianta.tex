\documentclass[a4paper,10pt]{article}

\usepackage[czech]{babel}
\usepackage[utf8]{inputenc}
\usepackage[T1]{fontenc}
\usepackage{geometry}
\usepackage{longtable}
\usepackage{minted}
\usepackage{xcolor}
\usepackage{hyperref}
\usepackage{titlesec}

\geometry{margin=1.5cm}
\setlength{\parindent}{0pt}
\setlength{\parskip}{0.3em}

% Vzhled sekcí
\titleformat{\section}{\Large\bfseries\sffamily}{\thesection}{1em}{}
\titleformat{\subsection}{\bfseries\sffamily}{\thesubsection}{1em}{}

% Barvy pro kód
\definecolor{bg}{rgb}{0.95,0.95,0.95}
\setminted{
    frame=none,
    bgcolor=bg,
    fontsize=\footnotesize,
    breaklines=true,
    baselinestretch=1.1
}

\begin{document}

\begin{center}
{\LARGE \textbf{Tahák základních příkazů Git (Bash)}}\\[4pt]
{\small Autor: Vaňha Jiří \quad | \quad \today}
\end{center}
\hrule
\vspace{1em}

\section*{Základní práce s repozitářem}
\begin{longtable}{p{0.38\linewidth}p{0.58\linewidth}}
\mintinline{bash}{git init} & Inicializuje nový lokální repozitář. \\
\mintinline{bash}{git clone <url>} & Naklonuje vzdálený repozitář (např. GitHub). \\
\mintinline{bash}{git status} & Zobrazí aktuální stav pracovního adresáře. \\
\mintinline{bash}{git add <soubor>} & Přidá konkrétní soubor do stagingu. \\
\mintinline{bash}{git add .} & Přidá všechny nové a změněné soubory. \\
\mintinline{bash}{git commit -m "popis"} & Uloží změny do historie s komentářem. \\
\mintinline{bash}{git log --oneline} & Stručný výpis commitů. \\
\mintinline{bash}{git diff} & Zobrazí rozdíly oproti poslednímu commitu. \\
\end{longtable}

\section*{Práce s větvemi}
\begin{longtable}{p{0.38\linewidth}p{0.58\linewidth}}
\mintinline{bash}{git branch} & Vypíše všechny lokální větve. \\
\mintinline{bash}{git branch <název>} & Vytvoří novou větev. \\
\mintinline{bash}{git checkout <název>} & Přepne se do dané větve. \\
\mintinline{bash}{git checkout -b <název>} & Vytvoří novou větev a přepne se do ní. \\
\mintinline{bash}{git merge <větev>} & Sloučí danou větev do aktuální. \\
\mintinline{bash}{git branch -d <větev>} & Smaže větev (pokud je sloučená). \\
\mintinline{bash}{git branch -D <větev>} & Násilně smaže větev (i když není sloučená). \\
\mintinline{bash}{git log master..bug --oneline} & Zobrazí commity, které jsou jen ve větvi \texttt{bug}. \\
\end{longtable}

\section*{Vzdálený repozitář (GitHub, GitLab)}
\begin{longtable}{p{0.38\linewidth}p{0.58\linewidth}}
\mintinline{bash}{git remote -v} & Zobrazí připojené vzdálené repozitáře. \\
\mintinline{bash}{git remote add origin <url>} & Přidá vzdálený repozitář. \\
\mintinline{bash}{git push -u origin master} & Nahraje větev \texttt{master} na server. \\
\mintinline{bash}{git pull} & Stáhne a sloučí změny z remote repozitáře. \\
\mintinline{bash}{git fetch} & Stáhne změny bez sloučení. \\
\mintinline{bash}{git push origin --delete <větev>} & Smaže vzdálenou větev. \\
\end{longtable}

\section*{Práce se změnami}
\begin{longtable}{p{0.38\linewidth}p{0.58\linewidth}}
\mintinline{bash}{git restore <soubor>} & Vrátí soubor do posledního commitu. \\
\mintinline{bash}{git restore --staged <soubor>} & Odebere soubor ze stagingu. \\
\mintinline{bash}{git reset --hard} & Vrátí vše do stavu posledního commitu (pozor!). \\
\mintinline{bash}{git stash push -m "záloha"} & Uloží rozpracované změny bokem. \\
\mintinline{bash}{git stash list} & Zobrazí uložené stash zálohy. \\
\mintinline{bash}{git stash apply stash@{0}} & Obnoví konkrétní stash. \\
\mintinline{bash}{git diff master..bug} & Porovná rozdíly mezi dvěma větvemi. \\
\end{longtable}

\section*{Patche a archivace}
\begin{longtable}{p{0.38\linewidth}p{0.58\linewidth}}
\mintinline{bash}{git diff master..bug > bug.patch} & Vytvoří patch soubor se změnami. \\
\mintinline{bash}{git apply bug.patch} & Aplikuje patch zpět do aktuální větve. \\
\mintinline{bash}{git format-patch master..bug -o ./patches} & Vytvoří sadu patchů pro každý commit. \\
\mintinline{bash}{git am ./patches/*.patch} & Aplikuje všechny patche i s historií. \\
\end{longtable}

\section*{Užitečné informace a diagnostika}
\begin{longtable}{p{0.38\linewidth}p{0.58\linewidth}}
\mintinline{bash}{git config --list} & Zobrazí aktuální konfiguraci. \\
\mintinline{bash}{git config user.name "Jméno"} & Nastaví jméno pro commity. \\
\mintinline{bash}{git config user.email "email@example.com"} & Nastaví e-mail. \\
\mintinline{bash}{git show <hash>} & Zobrazí detailní informace o commitu. \\
\mintinline{bash}{git reflog} & Ukáže i ztracené nebo přepsané commity. \\
\end{longtable}

\hrule
\bigskip
\textit{Poznámka:} Tento tahák obsahuje nejpoužívanější příkazy Gitu pro Bash.  
Pro pokročilou práci doporučuji prostudovat příkazy \mintinline{bash}{git rebase}, \mintinline{bash}{git cherry-pick}, \mintinline{bash}{git bisect} a \mintinline{bash}{git blame}.

\end{document}
