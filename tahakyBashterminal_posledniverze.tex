\documentclass[a4paper,10pt]{article}

\usepackage[czech]{babel}
\usepackage[utf8]{inputenc}
\usepackage[T1]{fontenc}
\usepackage{geometry}
\usepackage{longtable}
\usepackage{minted}
\usepackage{xcolor}
\usepackage{hyperref}
\usepackage{titlesec}

\geometry{margin=1.5cm}
\setlength{\parindent}{0pt}
\setlength{\parskip}{0.3em}

% Sekce a podsekce
\titleformat{\section}{\Large\bfseries\sffamily}{\thesection}{1em}{}
\titleformat{\subsection}{\bfseries\sffamily}{\thesubsection}{1em}{}

% Barvy pro kód
\definecolor{bg}{rgb}{0.95,0.95,0.95}
\setminted{
    frame=none,
    bgcolor=bg,
    fontsize=\footnotesize,
    breaklines=true,
    baselinestretch=1.1
}

\begin{document}

\begin{center}
{\LARGE \textbf{Git Super Tahák Expert}}\\[2pt]
{\small © 2025 ~Vaňha Jiří~(\textregistered). Distribuce povolena pro studijní účely.}
\end{center}
\hrule
\vspace{1em}

% =======================
\section*{1. Základní příkazy}
% =======================
\begin{longtable}{p{0.38\linewidth}p{0.58\linewidth}}
\mintinline{bash}{git init} & Inicializuje nový repozitář. \\
\mintinline{bash}{git clone <url>} & Naklonuje vzdálený repozitář. \\
\mintinline{bash}{git status} & Stav pracovního adresáře. \\
\mintinline{bash}{git add <soubor>} & Přidá soubor do stagingu. \\
\mintinline{bash}{git add .} & Přidá všechny změny. \\
\mintinline{bash}{git commit -m "popis"} & Uloží změny do historie. \\
\mintinline{bash}{git log --oneline} & Stručný výpis commitů. \\
\mintinline{bash}{git diff} & Rozdíly oproti poslednímu commitu. \\
\mintinline{bash}{git show <hash>} & Detail commitů. \\
\end{longtable}

% =======================
\section*{2. Větve}
% =======================
\begin{longtable}{p{0.38\linewidth}p{0.58\linewidth}}
\mintinline{bash}{git branch} & Zobrazí lokální větve. \\
\mintinline{bash}{git branch <název>} & Vytvoří novou větev. \\
\mintinline{bash}{git checkout <název>} & Přepne se do větve. \\
\mintinline{bash}{git checkout -b <název>} & Vytvoří a přepne se do větve. \\
\mintinline{bash}{git branch -m <nový>} & Přejmenuje aktuální větev. \\
\mintinline{bash}{git merge <větev>} & Sloučí větev. \\
\mintinline{bash}{git merge --abort} & Zruší probíhající merge. \\
\mintinline{bash}{git branch -d <větev>} & Smaže sloučenou větev. \\
\mintinline{bash}{git branch -D <větev>} & Násilné smazání větve. \\
\mintinline{bash}{git log master..bug --oneline} & Commity pouze ve větvi bug. \\
\mintinline{bash}{git cherry-pick <hash>} & Přidá commit z jiné větve. \\
\end{longtable}

% =======================
\section*{3. Vzdálené repozitáře}
% =======================
\begin{longtable}{p{0.38\linewidth}p{0.58\linewidth}}
\mintinline{bash}{git remote -v} & Zobrazí vzdálené repozitáře. \\
\mintinline{bash}{git remote add origin <url>} & Přidá remote repozitář. \\
\mintinline{bash}{git push -u origin master} & Nahraje master. \\
\mintinline{bash}{git pull} & Stáhne a sloučí změny. \\
\mintinline{bash}{git fetch} & Stáhne bez sloučení. \\
\mintinline{bash}{git push origin --delete <větev>} & Smaže vzdálenou větev. \\
\mintinline{bash}{git push origin --tags} & Nahraje všechny tagy. \\
\end{longtable}

% =======================
\section*{4. Práce se změnami a stash}
% =======================
\begin{longtable}{p{0.38\linewidth}p{0.58\linewidth}}
\mintinline{bash}{git restore <soubor>} & Vrátí soubor do posledního commitu. \\
\mintinline{bash}{git restore --staged <soubor>} & Odebere ze stagingu. \\
\mintinline{bash}{git reset --soft HEAD~1} & Vrátí poslední commit (ponechá změny). \\
\mintinline{bash}{git reset --hard HEAD~1} & Vrátí commit a odstraní změny. \\
\mintinline{bash}{git stash push -m "záloha"} & Uloží rozpracované změny bokem. \\
\mintinline{bash}{git stash list} & Zobrazí stash. \\
\mintinline{bash}{git stash apply stash@{0}} & Obnoví stash. \\
\mintinline{bash}{git diff master..bug} & Porovnání větví. \\
\end{longtable}

% =======================
\section*{5. Reverze a historie}
% =======================
\begin{longtable}{p{0.38\linewidth}p{0.58\linewidth}}
\mintinline{bash}{git revert <hash>} & Vytvoří commit vracející změny. \\
\mintinline{bash}{git reflog} & Historie pohybů HEAD. \\
\mintinline{bash}{git log --graph --oneline --all} & Přehled commitů s grafem. \\
\mintinline{bash}{git blame <soubor>} & Kdo upravil řádek souboru. \\
\mintinline{bash}{git shortlog -sn} & Statistika commitů podle autora. \\
\mintinline{bash}{git checkout <hash>} & Přepnutí do konkrétního commitu (detached HEAD). \\
\end{longtable}

% =======================
\section*{6. Tagy a verze}
% =======================
\begin{longtable}{p{0.38\linewidth}p{0.58\linewidth}}
\mintinline{bash}{git tag} & Zobrazí tagy. \\
\mintinline{bash}{git tag <název>} & Vytvoří lehký tag. \\
\mintinline{bash}{git tag -a v1.0 -m "verze"} & Anotovaný tag s popisem. \\
\mintinline{bash}{git show <tag>} & Zobrazí commit a info tagu. \\
\mintinline{bash}{git push origin <tag>} & Nahraje tag. \\
\mintinline{bash}{git tag -d <tag>} & Smaže tag. \\
\mintinline{bash}{git describe --tags --dirty} & Popis commit podle tagu. \\
\end{longtable}

% =======================
\section*{7. Pokročilé příkazy}
% =======================
\begin{longtable}{p{0.38\linewidth}p{0.58\linewidth}}
\mintinline{bash}{git log -p} & Zobrazí diff u každého commitu. \\
\mintinline{bash}{git log --stat} & Přehled změněných souborů. \\
\mintinline{bash}{git log --author="Šéfe"} & Commity od autora. \\
\mintinline{bash}{git log --grep="fix"} & Vyhledání v message. \\
\mintinline{bash}{git log --since="2 weeks ago"} & Commity za poslední 14 dní. \\
\mintinline{bash}{git diff --name-only <hash1> <hash2>} & Jen názvy změněných souborů. \\
\mintinline{bash}{git add -p} & Přidávání změn po částech. \\
\mintinline{bash}{git restore -p} & Obnova změn po částech. \\
\mintinline{bash}{git checkout <commit> -- <soubor>} & Obnova souboru z minulého commitu. \\
\mintinline{bash}{git rebase -i HEAD~3} & Interaktivní rebase posledních 3 commitů. \\
\mintinline{bash}{git commit --amend} & Úprava posledního commitu. \\
\mintinline{bash}{git clean -n/f} & Náhled/smazání nesledovaných souborů. \\
\mintinline{bash}{git mv <starý> <nový>} & Přejmenování/přesun souboru. \\
\mintinline{bash}{git rm <soubor>} & Smazání souboru. \\
\mintinline{bash}{git grep <text>} & Hledání textu v repozitáři. \\
\mintinline{bash}{git archive --format=zip --output=v1.1.zip HEAD} & ZIP aktuální verze. \\
\mintinline{bash}{git bisect start/good/bad/reset} & Binární hledání chyby. \\
\mintinline{bash}{git config --global alias.st status} & Alias pro git st. \\
\mintinline{bash}{git config --global alias.lg "log --oneline --graph --decorate --all"} & Krásný log. \\
\mintinline{bash}{git config --global alias.unstage "reset HEAD --"} & Odebrání ze stagingu. \\
\mintinline{bash}{git remote show origin} & Přehled vzdáleného repozitáře. \\
\mintinline{bash}{git fetch --prune} & Stáhne a odstraní smazané větve. \\
\mintinline{bash}{git branch -vv} & Která větev sleduje kterou remote. \\
\mintinline{bash}{git push --force-with-lease} & Bezpečné přepsání remote větve. \\
\mintinline{bash}{git count-objects -vH} & Statistiky velikosti repozitáře. \\
\mintinline{bash}{git rev-parse HEAD} & Hash aktuálního commitu. \\
\mintinline{bash}{git verify-pack -v .git/objects/pack/*.idx} & Analýza velikosti objektů. \\
\mintinline{bash}{git help <příkaz>} & Nápověda pro příkaz. \\
\mintinline{bash}{git <příkaz> --help} & Totéž. \\
\mintinline{bash}{git config --help} & Nápověda konfigurace. \\
\end{longtable}

% =======================
\section*{8. Interní Git kódy a reference}
% =======================

\subsection*{1) Interní reference a hashy}
\begin{longtable}{p{0.38\linewidth}p{0.58\linewidth}}
\mintinline{bash}{HEAD} & Aktuální commit. \\
\mintinline{bash}{HEAD~1 / HEAD~3} & Jeden / tři commity zpět. \\
\mintinline{bash}{HEAD^ / HEAD^2} & Rodiče merge commitu. \\
\mintinline{bash}{<hash>} & Plný nebo zkrácený commit hash. \\
\mintinline{bash}{ORIG_HEAD} & Poslední stav HEAD před resetem/merge. \\
\mintinline{bash}{FETCH_HEAD} & Poslední stažený commit z fetch. \\
\mintinline{bash}{MERGE_HEAD} & Merge HEAD větve. \\
\mintinline{bash}{refs/stash} & Poslední stash. \\
\end{longtable}

\subsection*{2) Speciální stavy a módy}
\begin{longtable}{p{0.38\linewidth}p{0.58\linewidth}}
\mintinline{bash}{detached HEAD} & Stav mimo větev. \\
\mintinline{bash}{--amend} & Úprava posledního commitu. \\
\mintinline{bash}{--force-with-lease} & Bezpečné přepsání vzdálené větve. \\
\mintinline{bash}{--hard / --soft / --mixed} & Typ resetu. \\
\mintinline{bash}{--patch / -p} & Interaktivní výběr změn. \\
\mintinline{bash}{--stat / -p} & Změny u logu. \\
\mintinline{bash}{--oneline / --graph / --decorate} & Přehledný log s grafem. \\
\end{longtable}

\subsection*{3) Barevné a formátovací kódy}
\begin{longtable}{p{0.38\linewidth}p{0.58\linewidth}}
\mintinline{bash}{--color=auto} & Barevný výpis. \\
\mintinline{bash}{%h} & Zkrácený hash. \\
\mintinline{bash}{%H} & Plný hash. \\
\mintinline{bash}{%an / %ae} & Autor jméno / email. \\
\mintinline{bash}{%s} & Commit message. \\
\mintinline{bash}{%d} & Dekorace (tagy, větve). \\
\mintinline{bash}{%cr / %cd} & Relativní čas / datum. \\
\mintinline{bash}{git log --pretty=format:"%h %an %cr - %s"} & Příklad výpisu. \\
\end{longtable}

\subsection*{4) Git objekty a nízkoúrovňové příkazy}
\begin{longtable}{p{0.38\linewidth}p{0.58\linewidth}}
\mintinline{bash}{git cat-file -t <hash>} & Typ objektu (commit, tree, blob, tag). \\
\mintinline{bash}{git cat-file -p <hash>} & Obsah objektu. \\
\mintinline{bash}{git ls-tree <tree>} & Obsah stromu (adresáře). \\
\mintinline{bash}{git rev-list --objects HEAD} & Seznam objektů. \\
\mintinline{bash}{git hash-object <soubor>} & SHA-1 hash souboru. \\
\mintinline{bash}{git update-index --assume-unchanged <soubor>} & Ignorovat lokální změny. \\
\end{longtable}

\hrule
\bigskip
\textit{Poznámka:} Tento Super Tahák kombinuje všechny základní, pokročilé příkazy a interní Git kódy pro maximální efektivitu práce.

\end{document}
